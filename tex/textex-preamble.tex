\usepackage[utf8]{luainputenc}
\usepackage{tikz}
\usepackage{amsmath, commath, mathtools}
\usepackage{lualatex-math}
\usepackage{xcolor}


%%% Font selection and related definitions
%====================================================================

% Mostly imported from tLibertinus
\usepackage{pgfornament}
\usepackage{fontspec}
\defaultfontfeatures{RawFeature=+calt}
\setmainfont{libertinusserif-regular.otf}[
    Path = {/home/timtro/.fonts/type1/Libertinus/}
  , ItalicFont = libertinusserif-italic.otf
  , BoldFont = libertinusserif-semibold.otf
  , BoldItalicFont = libertinusserif-bolditalic.otf ]
\setsansfont{libertinussans-regular.otf}[
    Path = {/home/timtro/.fonts/type1/Libertinus/}
  , ItalicFont = libertinussans-italic.otf
  , BoldFont = libertinussans-bold.otf ]
\setmonofont{Hasklig}[
    Scale = MatchLowercase
  , BoldFont = *-Semibold
  , ItalicFont = *-It
  , BoldItalicFont = *-SemiboldIt ]

\newfontfamily{\textdf}[
  Path = {/home/timtro/.fonts/type1/Libertinus/}
    ]{libertinusserifdisplay-regular.otf}

\newfontfamily{\textsb}[
  Path = {/home/timtro/.fonts/type1/Libertinus/}
    ]{libertinusserif-semibold.otf}

\newfontfamily{\textsbit}[
  Path = {/home/timtro/.fonts/type1/Libertinus/}
    ]{libertinusserif-semibolditalic.otf}

\ifluatex\usepackage{lualatex-math}\fi
\usepackage[math-style=ISO, partial=upright]{unicode-math}
\setmathfont[
    Path = {/home/timtro/.fonts/type1/Libertinus/}
  , AutoFakeBold
    ]{libertinusmath-regular.otf}

\setmathfont[range={}]{libertinusmath-regular.otf} % Reload with empty range for correct spacing

\renewcommand{\familydefault}{\sfdefault}

%%% Drawing and Diagrams
%====================================================================

\usepackage{tikz-cd}
\usetikzlibrary{arrows,arrows.meta,positioning,shapes,snakes,fit,calc}
\pgfdeclarelayer{background}
\pgfsetlayers{background,main}
\tikzcdset{arrow style=tikz, diagrams={>=stealth'}}
\tikzset{ shorten <>/.style={ shorten >=#1, shorten <=#1 } }
% \usepackage{tikz-uml}
% \tikzumlset{font=\footnotesize\ttfamily}
% \tikzumlset{fill class=white}

%%% Code listings
%====================================================================
\usepackage{tCodeListing}
